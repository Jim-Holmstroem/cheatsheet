\documentclass[10pt,landscape]{article}
\usepackage{multicol}
\usepackage{calc}
\usepackage{ifthen}
\usepackage{hyperref}
\usepackage[landscape]{geometry}
\usepackage{xstring}
\usepackage{amsmath}

% This sets page margins to .5 inch if using letter paper, and to 1cm
% if using A4 paper. (This probably isn't strictly necessary.)
% If using another size paper, use default 1cm margins.
\ifthenelse{\lengthtest { \paperwidth = 11in}}
	{ \geometry{top=.5in,left=.5in,right=.5in,bottom=.5in} }
	{\ifthenelse{ \lengthtest{ \paperwidth = 297mm}}
		{\geometry{top=1cm,left=1cm,right=1cm,bottom=1cm} }
		{\geometry{top=1cm,left=1cm,right=1cm,bottom=1cm} }
	}

% Turn off header and footer
\pagestyle{empty}

% Redefine section commands to use less space
\makeatletter
\renewcommand{\section}{\@startsection{section}{1}{0mm}%
                                {-1ex plus -.5ex minus -.2ex}%
                                {0.5ex plus .2ex}%x
                                {\normalfont\large\bfseries}}
\renewcommand{\subsection}{\@startsection{subsection}{2}{0mm}%
                                {-1explus -.5ex minus -.2ex}%
                                {0.5ex plus .2ex}%
                                {\normalfont\normalsize\bfseries}}
\renewcommand{\subsubsection}{\@startsection{subsubsection}{3}{0mm}%
                                {-1ex plus -.5ex minus -.2ex}%
                                {1ex plus .2ex}%
                                {\normalfont\small\bfseries}}
\makeatother

% Define BibTeX command
\def\BibTeX{{\rm B\kern-.05em{\sc i\kern-.025em b}\kern-.08em
    T\kern-.1667em\lower.7ex\hbox{E}\kern-.125emX}}

% Don't print section numbers
\setcounter{secnumdepth}{0}

\setlength{\parindent}{0pt}
\setlength{\parskip}{0pt plus 0.5ex}

\newcommand{\Subject}{Thermal Physics}

\hypersetup
{
    pdfauthor = {Jim Holmstr\"{o}m},
    pdfsubject = {\Subject{}},
    pdftitle = {\Subject{} Cheat Sheet},
    pdfkeywords = {\Subject{}}
}

\newcommand{\ts}{\textsuperscript}
\newcommand{\refbook}[1]{\ts{\tiny{#1}}}
\newcommand{\dd}[2]{\frac{\partial {#1}}{\partial {#2}}}

% -----------------------------------------------------------------------

\begin{document}

\raggedright
\footnotesize
\begin{multicols}{3}


% multicol parameters
% These lengths are set only within the two main columns
%\setlength{\columnseprule}{0.25pt}
\setlength{\premulticols}{1pt}
\setlength{\postmulticols}{1pt}
\setlength{\multicolsep}{1pt}
\setlength{\columnsep}{2pt}

\newlength{\TabularLen}
\settowidth{\TabularLen}{\texttt{letterpaper}/\texttt{a4paper} \ }

%Start here
\begin{center}
    \Large{\textbf{\Subject{} Cheat Sheet}} \\
\end{center}

%Remove top-sections and subsection->section?
\section{Thermodynamics}
    \subsection{Basics\refbook{1-3}}
    \subsection{Temperature \& Boltzmann Factor\refbook{4}}
    \subsection{Maxwell-Boltzmann Distribution\refbook{5}}
    \subsection{Pressure \& Ideal Gas Law\refbook{6}}
    \subsection{Molecular Flux \& Effusion\refbook{7}}
    \subsection{Mean Free Path \& Collisions\refbook{8}}
    \subsection{Energy\refbook{11}}
    \subsection{Adiabatic Processes\refbook{12}}
    \subsection{Heat Engine 2\ts{nd} Law\refbook{13}}
    \subsection{Entropy\refbook{14}}
    \subsection{Thermodynamic Potentials\refbook{16}}
        \subsubsection{Internal energy, U}
            \begin{equation}
                \label{eq:internalEnergy}
                dU 
                = TdS - pdV
            \end{equation}
        \subsubsection{Enthalpy, H}
            \begin{equation}
                H 
                \equiv U + PV 
                = \left\{\eqref{eq:dHDerive}\right\}
                = H(S,p)
            \end{equation}
            \begin{equation}
                \label{eq:dHDerive}
                dH 
                = \left\{\eqref{eq:internalEnergy}\right\}
                = TdS - pdV + pdV + VdP 
                = TdS + Vdp
            \end{equation}
            \begin{equation}
                \Delta H
                = \begin{cases}
                    exothermic & \Delta H < 0 \\
                    endothermic& \Delta H > 0 \\
                \end{cases}
            \end{equation}
        \subsubsection{Helmholtz function, F}
            \begin{equation}
                F 
                \equiv U - TS
                = \left\{\eqref{eq:dFDerive}\right\}
                = F(T,V)
            \end{equation}
            \begin{equation}
                \label{eq:dFDerive}
                dF
                = \left\{\eqref{eq:internalEnergy}\right\}
                = TdS - pdV - TdS - SdT
                = -SdT - pdV
            \end{equation}
        \subsubsection{Gibbs function, G}
            \begin{equation}
                G 
                \equiv H - TS
                = \left\{\eqref{eq:dGDerive}\right\}
                = G(T,p)
            \end{equation}
            \begin{equation}
                \label{eq:dGDerive}
                dG
                = \left\{\eqref{eq:dHDerive}\right\}
                = TdS + VdP - TdS - SdT
                = -SdT + VdP
            \end{equation}

    \subsection{Maxwell Relations\refbook{16}}
        Derivation of generalized maxwell
        \begin{equation}
            df\left(x,y\right) 
            = \left(\dd{f\left(x,y\right)}{x}\right)_y dx 
            + \left(\dd{f\left(x,y\right)}{y}\right)_x dx 
        \end{equation}

    \subsection{Work Generalization\refbook{17}}
    \subsection{3\ts{rd} Law\refbook{18}}
\section{Classical Statistical Mechanics}
    \subsection{Equipartition\refbook{19}}
    \subsection{Partition Function\refbook{20}}
    \subsection{Statistical Mechanics on Ideal Gases\refbook{21}}
    \subsection{Chemical Potential\refbook{22}}
\section{Quantum statistics}
    \subsection{Bose-Einstein Distribution\refbook{29}}
    \subsection{Bose Gases\refbook{30}}
    \subsection{Fermi-Dirac Distribution\refbook{29}}
    \subsection{Fermi Gases\refbook{30}}
    \subsection{Phonons\refbook{23,34}}
    \subsection{Real Gases\refbook{26.1,26.4}}
    \subsection{Phase Transisions\refbook{28.1-3}}

Copyright \copyright\ 2013 Jim Holmstr\"{o}m

%http://tex.stackexchange.com/questions/107541/lowercase-inside-url/
\newcommand\lcURL[1]{%
  \begingroup
  \StrSubstitute{#1}{ }{}[\tmp]%
    \edef\tmp{%
      \lowercase{%
        \endgroup
        \noexpand\url{http://www.cheatsheet.jim.pm?subject=\tmp}%
      }%
    }%
  \tmp
}

\lcURL{\Subject}

\end{multicols}
\end{document}
